%!TEX TS-program = xelatex
%!TEX options = -aux-directory=Debug -shell-escape -file-line-error -interaction=nonstopmode -halt-on-error -synctex=1 "%DOC%"
\documentclass{article}
\input{LaTeX-Submodule/template.tex}

% Additional packages & macros

% Header and footer
\newcommand{\unitName}{Computational Mathematics 2}
\newcommand{\unitTime}{Semester 1, 2024}
\newcommand{\unitCoordinator}{Dr Elliot Carr}
\newcommand{\documentAuthors}{Tarang Janawalkar}

\fancyhead[L]{\unitName}
\fancyhead[R]{\leftmark}
\fancyfoot[C]{\thepage}

% Copyright
\usepackage[
    type={CC},
    modifier={by-nc-sa},
    version={4.0},
    imagewidth={5em},
    hyphenation={raggedright}
]{doclicense}

\date{}

\begin{document}
%
\begin{titlepage}
    \vspace*{\fill}
    \begin{center}
        \LARGE{\textbf{\unitName}} \\[0.1in]
        \normalsize{\unitTime} \\[0.2in]
        \normalsize\textit{\unitCoordinator} \\[0.2in]
        \documentAuthors
    \end{center}
    \vspace*{\fill}
    \doclicenseThis
    \thispagestyle{empty}
\end{titlepage}
\newpage
%
\tableofcontents
\newpage
%
\section{Finite Volume Methods}
\subsection{Transport Phenomena}
Transport phenomena broadly comprises three disciplines; fluid
dynamics, heat transfer, and mass transfer.\ \textbf{Fluid dynamics} is
the study of the motion of fluids, including liquids and gases.\
\textbf{Heat transfer} is the study of how heat (thermal energy) is
transported, generated, dissipated, and/or converted in a physical
system.\ \textbf{Mass transfer} is the study of the movement of mass
from one location to another.

The mathematical equations used to describe the above phenomena involve
three fundamental mechanisms of transport:
\begin{enumerate}
    \item Diffusion
    \item Advection
    \item Reaction
\end{enumerate}
\subsubsection{Diffusion}
\textit{Diffusion} is the gradual movement of a substance from regions of high
concentration to regions of low concentration. The direction of
diffusion is determined by the sign of the negative gradient of the
concentration.
\subsubsection{Advection}
\textit{Advection} is the transport of a substance by bulk motion of a fluid.
Advection is driven by a vector field in which the substance is
transported.
\subsubsection{Reaction}
Reaction is the process in which substances are created or destroyed.
Reaction is represented as a source or sink function.
\subsection{Transport Equation}
The general form of the transport equation is given by
\begin{equation*}
    \underbrace{\pdv{u}{t}}_{\text{unsteady term}} + \underbrace{\symbf{\nabla} \cdot \left( \symbf{v} u \right) \vphantom{\pdv{u}{t}}}_{\text{advection term}} = \underbrace{\symbf{\nabla} \cdot \left( \symbf{D} \symbf{\nabla} u \right) \vphantom{\pdv{u}{t}}}_{\text{diffusion term}} + \underbrace{R \vphantom{\pdv{u}{t}}}_{\text{reaction term}}
\end{equation*}
where \(u\left( \symbf{x},\: t \right)\) is the quantity being transported
at position \(\symbf{x}\) and time \(t\), \(\symbf{v}\left( \symbf{x},\: t \right) \in \R^n\)
is a velocity vector field which drives \(u\),
\(\symbf{D} \in \R^{n \times n}\) is the diffusion matrix, and \(R\) is
a reaction term.\ \(n\) represents the dimension of the spatial domain
of the problem, which can be 1, 2, or 3.

An alternative form of the transport equation combines the divergence
terms
\begin{equation*}
    \underbrace{\pdv{u}{t} \vphantom{\pdv{u}{t}}}_{\text{unsteady term}} + \underbrace{\symbf{\nabla} \cdot \symbf{q} \vphantom{\pdv{u}{t}}}_{\text{flux term}} = \underbrace{R \vphantom{\pdv{u}{t}}}_{\text{reaction term}}
\end{equation*}
where \(\symbf{q} = \symbf{v} u - \symbf{D} \symbf{\nabla} u\) is the
\textit{flux vector}.
\subsubsection{Domain}
This PDE is defined on a specified domain \(\Omega\) which is an open
connected subset of \(\R^n\), with the boundary \(\partial \Omega\).
\subsubsection{Derivation}
The transport equation can be derived from the conservation of mass
principle: the rate of change of the quantity \(u\) within a region
\(D\) must be balanced by the net flow of \(u\) in/out of the boundary
\(\partial{D}\) of \(D\), and the rate of creation or destruction of
\(u\) within \(D\). Consider an arbitrarily small sub-domain \(D\) of
\(\Omega\) with boundary \(\partial D\), then:
\begin{align*}
    \biggl\{ \parbox[c]{2.5cm}{\centering Rate of change                                                                                                          \\ of \(u\) in \(D\)} \biggr\} & =
    -\biggl\{ \parbox[c]{2cm}{\centering \(u\) leaving \(D\)                                                                                                      \\ across \(\partial{D}\)} \biggr\}
    + \biggl\{ \parbox[c]{4cm}{\centering Generation/Destruction                                                                                                  \\of \(u\) within \(D\)} \biggr\} \\
    \odv{}{t} \int_D u \odif{V}                                                    & = -\int_{\partial{D}} \symbf{q} \cdot \symbf{n} \odif{s} + \int_D R \odif{V} \\
    \int_D \pdv{u}{t} \odif{V}                                                     & = -\int_{D} \symbf{\nabla} \cdot \symbf{q} \odif{V} + \int_D R \odif{V}      \\
    \int_D \left( \pdv{u}{t} + \symbf{\nabla} \cdot \symbf{q} - R \right) \odif{V} & = 0                                                                          \\
    \pdv{u}{t} + \symbf{\nabla} \cdot \symbf{q}                                    & = R.
\end{align*}
\subsection{Special Cases}
\subsubsection{One Spatial Dimension}
In one spatial dimension, the transport equation reduces to
\begin{equation*}
    \pdv{u}{t} + \pdv{}{x} \left( v u \right) = \pdv{}{x} \left( D \pdv{u}{x} \right) + R.
\end{equation*}
where \(u\left( x,\: t \right)\) is a function of one spatial dimension
and time, \(v\) is the velocity, and \(D > 0\) is the diffusivity.
\subsubsection{Two Spatial Dimensions}
In two spatial dimensions, the transport equation reduces to
\begin{equation*}
    \pdv{u}{t} + \pdv{}{x} \left( v_x u \right) + \pdv{}{y} \left( v_y u \right) = \pdv{}{x} \left( D_{xx} \pdv{u}{x} \right) + \pdv{}{y} \left( D_{yy} \pdv{u}{y} \right) + R.
\end{equation*}
where \(u\left( x,\: y,\: t \right)\) is a function of two spatial
dimensions and time, \(v_x\) and \(v_y\) are the velocities in the \(x\)
and \(y\) directions, and \(D_{xx}\) and \(D_{yy}\) are the diffusivities
in the \(x\) and \(y\) directions.
\subsubsection{Eliminating Terms}
The transport equation is also called the
\textit{advection-diffusion-reaction equation}.
\begin{itemize}
    \item If the \textit{velocity term} \(\symbf{v}\) is the zero
          vector, the equation reduces to the
          \textit{diffusion-reaction equation}.
    \item If the \textit{diffusion term} \(\symbf{D}\) is the zero
          matrix, the equation reduces to the
          \textit{advection-reaction equation}.
    \item If the \textit{reaction term} \(R\) is zero, the equation
          reduces to the \textit{advection-diffusion equation}.
\end{itemize}
\subsection{Classification}
While the terms in the transport equation may be constant or variable,
certain combinations of these terms lead to different solution methods.
\begin{itemize}
    \item The velocity vector \(\symbf{v}\) may be a constant vector or
          a function of space \(\symbf{x}\), time \(t\), and/or the
          solution \(u\).
    \item The diffusion matrix \(\symbf{D}\) may be a constant matrix
          or a function of space \(\symbf{x}\), time \(t\), and/or the
          solution \(u\).
    \item The reaction term \(R\) may be a constant or a function of
          space \(\symbf{x}\), time \(t\), and/or the solution \(u\).
\end{itemize}
When \(\symbf{v}\) and \(\symbf{D}\) are \underline{not} functions of \(u\), and
\(R\) is a \underline{linear} function of \(u\), the transport equation is
called \textit{linear}. The equation is \textit{nonlinear} otherwise.
The domain \(\Omega\) is called \textit{heterogeneous} if any of the
coefficients \(\symbf{v}\), \(\symbf{D}\), or \(R\) are functions of
space \(\symbf{x}\), and \textit{homogeneous} otherwise.
\subsection{Dimensional Analysis}
Performing a dimensional analysis on the transport equation allows us
to associate physical units with the coefficients of the equation. This
analysis is useful for verifying the correctness of the equation and
for scaling the equation to a dimensionless form. The terms in the
equation
\begin{equation*}
    \pdv{u}{t} + \symbf{\nabla} \cdot \left( \symbf{v} u \right) = \symbf{\nabla} \cdot \left( \symbf{D} \symbf{\nabla} u \right) + R
\end{equation*}
may only be added or subtracted if they have the same units. Therefore,
given that
\begin{equation*}
    \left[ \pdv{u}{t} \right] \equiv \frac{\left[ u \right]}{\left[ t \right]} = \frac{\left[ u \right]}{\mathsf{T}}
\end{equation*}
we can deduce the units of other terms in the equation.
\begin{align*}
    \left[ \symbf{\nabla} \cdot \left( \symbf{v} u \right) \right] \equiv \frac{\left[ \symbf{v} \right] \left[ u \right]}{\left[ x \right]} = \frac{\left[ u \right]}{\mathsf{T}}                                                                                                        & \implies \left[ \symbf{v} \right] = \frac{\mathsf{L}}{\mathsf{T}}   \\
    \left[ \symbf{\nabla} \cdot \left( \symbf{D} \symbf{\nabla} u \right) \right] \equiv \frac{\left[ \symbf{D} \right] \left[ \symbf{\nabla} u \right]}{\left[ x \right]} = \frac{\left[ \symbf{D} \right] \left[ u \right]}{{\left[ x \right]}^2} = \frac{\left[ u \right]}{\mathsf{T}} & \implies \left[ \symbf{D} \right] = \frac{\mathsf{L}^2}{\mathsf{T}} \\
    \left[ R \right] \equiv \frac{\left[ u \right]}{\left[ t \right]}                                                                                                                                                                                                                     & \implies \left[ R \right] = \frac{\left[ u \right]}{\mathsf{T}}
\end{align*}
\subsection{Initial and Boundary Conditions}
In addition to the transport equation, which describes the behaviour of
\(u\) within the domain \(\Omega\), the problem must also specify how
\(u\) behaves at the boundary \(\partial \Omega\) with \textit{boundary
conditions}. Some common boundary conditions include:
\begin{itemize}
    \item Specified value: \(u\left( \symbf{x},\: t \right) = u_b\) on
          \(\partial \Omega\)
    \item Specified flux: \(\symbf{q} \cdot \symbf{n} = q_b\) on
          \(\partial \Omega\)
    \item Specified gradient: \(\symbf{\nabla} u \cdot \symbf{n} =
          d_b\) on \(\partial \Omega\)
\end{itemize}
Here \(u_b\), \(q_b\), and \(d_b\) may be constants or scalar functions
of \(\symbf{x}\) and/or \(t\), and \(\symbf{n}\) is the unit normal vector
to \(\partial \Omega\), directed outward from \(\Omega\).

We may also wish to use a Robin condition to describe a general boundary
condition of the form:
\begin{equation*}
    a u + b \left( \symbf{\nabla} u \cdot \symbf{n} \right) = c
\end{equation*}
where \(a\), \(b\), and \(c\) are constants or scalar functions of
\(\symbf{x}\) and/or \(t\). When \(c = 0\), the condition is called
\textit{homogeneous}, and \textit{nonhomogeneous} otherwise.

In addition to these conditions, an \textit{initial condition} is required to
specify the profile of \(u\) at time \(t = 0\).
\subsection{Steady-State Problems}
If it exists, the \textit{steady-state solution} of the transport equation
is the solution of the equation when the time-derivative of \(u\) is zero:
\begin{equation*}
    \pdv{u}{t} = 0.
\end{equation*}
The steady-state solution is useful for understanding the long-term
behaviour of the system, where it is assumed that the system is no longer
time-dependent. The steady-state solution is expressed as
\(u_\infty = \lim_{t \to \infty} u\left( \symbf{x},\: t \right)\).
\end{document}
